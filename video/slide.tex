\documentclass[11pt]{beamer}
\usetheme{Warsaw}
\usepackage[utf8]{inputenc}
\usepackage{natbib}
\usepackage{amsmath}
\usepackage{graphicx}
\usepackage{subcaption}
\usepackage{tikz}


\title{UE COM}
\subtitle{Learning Deep CNN Denoiser Prior for Image Restoration}
\author{Adrien Zabban}
\date{8 janvier 2024}

\begin{document}

\maketitle

% --- Probleme ---
\begin{frame}{Le Probleme inverse}
    \begin{block}{But}
        On a une image observée dégradée $y$ et l'on veut retrouver l'image d'origine $x$. On sait 
        que cette image a été dégradée de la façon suivante : 
        $$y = Hx + v$$
        où $H$ est la matrice de 
        dégradation que l'on connait, et $v$ est un bruit gaussien d'écart-type $\sigma$ inconnue.
    \end{block}
    \begin{figure}[b]
        \centering
        \includegraphics[width=0.41\textwidth]{../debluring/x.png}
        \begin{tikzpicture}[overlay, remember picture]
            \draw[->, blue, very thick] (0,1.5) -- (1,1.5) node[midway, above, blue] {$f_{H, \sigma}$};
        \end{tikzpicture}
        \hspace{1cm}
        \includegraphics[width=0.41\textwidth]{../debluring/y.png}
        \caption{image d'orignine x (à gauche) et l'image dégradée y (à droite).}
    \end{figure}
\end{frame}


\begin{frame}{Maximiser la log likelihood}
    \begin{visibleenv}<1->
        \begin{align*}
            \max_x \log(p(x|y)) &= \max_x \log(p(x, y)) \quad \text{car } p(x|y) = p(x, y) \times p(y) \\
            &= \max_x \log(p(y|x)) + \log(p(x)) \\
            &\quad \text{or } (y|x) = (v+Hx|x) \sim \mathcal{N}(Hx, \sigma^2) \\
            &= \max_x -\frac{||y-Hx||^2}{2 \sigma^2} + \log(p(x)) \\
            &= \min_x \frac{1}{2}||y-Hx||^2 + \lambda \Phi(x) \quad \text{avec } \Phi = -\frac{\log \circ p}{\lambda}
        \end{align*}
    \end{visibleenv}

    \begin{visibleenv}<2->
        \begin{alertblock}{But}
            On veut donc trouver $\hat{x}$ tel que: $\hat{x} = \text{arg} \min_x \frac{1}{2}||y-Hx||^2 + \lambda \Phi(x)$
        \end{alertblock}
    \end{visibleenv}
\end{frame}

\begin{frame}{Une première méthode: ISTA}
    raconter ISTA
\end{frame}

\begin{frame}{Une deuxièm méthode: HQS}
    raconter HQS
\end{frame}

\begin{frame}{Les systèmes de plug and play}
    en quoi ça consiste
\end{frame}

\begin{frame}{Le Denoiser}
    le model
\end{frame}

\begin{frame}{Le Denoiser}
    train et inférance
\end{frame}

\begin{frame}{Plug and play}
    résultats
\end{frame}



\end{document}
